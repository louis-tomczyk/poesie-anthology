% ====================================================================== %
%   author :   louis tomczyk
%   date   :   2024.10.01
%   version:   1.0.0
% ====================================================================== %
%   ChangeLog
%   (1.0.0) - 2024.10.01   : création
% ====================================================================== %
\myPageBreak

% ====================================================================== %
\poem{1}{Sur la mort de Marie}{\isPayant}{
comme on voit sur la branche au mois de mai la rose,\\
en sa belle jeunesse, en sa première fleur,\\
rendre le ciel jaloux de sa vive couleur,\\
quand l'aube de ses pleurs au point du jour l'arrose.\\
\\
la graĉe dans sa feuille et l'amour se repose\\
embaumant le jardin et les arbres d'odeurs\\
mais batttue ou de pluis ou d'excessive ardeur\\
languissante elle meurt, feuille à feuille déclose.\\
\\
ainsi en ta première et jeune nouveauté\\
quand la terre et le ciel honoraient ta beauté\\
la Parque t'a tuée, et cendres tu reposes.\\
\\
pour obsèques reçois mes larmes et mes pleurs,\\
ce vase plein de lait, ce panier plein de fleurs\\
afin que vif et mort ton corps ne soit que roses.
}{Pierre de Ronsard, 1578}

% ====================================================================== %
\poemBreakCover{2}{Le pont Mirabeau}{\isPayant}{
sur le pont Mirabeau coule la Seine\\
et nos amours faut-il que je m'en souvienne\\
la joie venait toujours après la peine\\
\\
\tabto{2cm} vienne la nuit, sonne l'heure\\
\tabto{2cm} les jours s'en vont, je demeure\\
\\
les mains dans les mains, restons face à face\\
tandis que sous le pont de nos bras passe\\
des éternels regards, l'onde si lasse\\
\\
\tabto{2cm} vienne la nuit, sonne l'heure\\
\tabto{2cm} les jours s'en vont, je demeure\\
\\
l'amour s'en va comme cette eau courante\\
l'amour s'en va, comme la vie est lente\\
et comme l'espérance est violente\\
\\
\tabto{2cm} vienne la nuit, sonne l'heure\\
\tabto{2cm} les jours s'en vont, je demeure\\
}{Guillaume Apollinaire, Les soirées de Paris - Alcools, 1913}

\poemBreakEnd{
passant les jours et les semaines\\
ni temps passé, ni les amours reviennent\\
sous le pont Mirabeau coule la Seine.\\
\\
\tabto{2cm} vienne la nuit, sonne l'heure\\
\tabto{2cm} les jours s'en vont, je demeure
}

% ====================================================================== %
\poem{3}{Correspondances}{\isPayant}{
la nature est un temple où de vivants pilliers\\
laissent parfois sortir de confuses paroles\\
l'homme y passe à travers des forêts de symboles\\
qui l'observent avec des regards familiers\\
\\
comme de longs échos qui de loin se confondent\\
dans une ténébreuse et profonde unité\\
vaste comme la nuit et comme la clarté\\
les parfums, les couleurs et les sons se répondent\\
\\
il est des parfums frais comme des chairs d'enfants\\
doux comme des hautbois, verts comme les prairies\\
et d'autres, corrompus, riches et triomphants\\
\\
ayant l'expansion des choses infinies\\
comme l'ambre, le musc, le benjoin et l'encens\\
qui chantent les transports de l'esprit et des sens.
}{Charles Baudelaire, Les fleurs du mal, 1857}


% ====================================================================== %
\poem{3}{À une passante}{\isPayant}{
la rue assourdissante autour de moi hurlait\\
longue, mince, en grand deuil, douleur majestueuse,\\
une femme passa d'une main fastueuse\\
soulevant, balançant le feston et l'ourlet.\\
\\
agile et noble, avec sa jambde de statue.\\
moi je buvaus, crispé comme un extravaguant,\\
dans son \oe{}il, ciel livide où germe l'ouragan,\\
la douceur qui fascine et le plaisir qui tue.\\
\\
un éclair... puis la nuit! fugitive beauté\\
dont le regard m'a fait soudainement renaître\\
ne te reverrai-je plus que dans l'éternité?\\
\\
ailleurs, bien loin d'ici! trop tard! jamais peut-être!\\
car j'ignore où tu fuis, tu ne sais où je vais,\\
ô toi que j'eusse aimé, ô toi qui le savais!
}{}

% ====================================================================== %
\poemBreakCover{3}{l'invitation au voyage}{\isPayant}{
mon enfant, ma s\oe{}ur,\\
songe à la douceur\\
d'aller là-bas vivre ensemble!\\
aimer à loisir\\
aimer et mourir\\
au pays qui te ressemble!\\
\\
les soleils mouillés\\
de ces ciels brouillés\\
pour mon esprit ont les charmes\\
si mystérieux\\
de tes traites yeux\\
brillants au travers leurs larmes\\
\\
\tabto{2cm} là, tout n'est qu'ordre et beauté\\
\tabto{2cm} luxe, calme et volupté\\
\\
des meubles luisants\\
polis par les ans\\
décoreraietn notre chambre\\
les plus rares fleurs\\
mêlant leurs odeurs\\
aux vagues senteurs de l'ambre\\
}{}

\poemBreakEnd{
les riches plafonds\\
les miroirs profonds\\
la splendeur orientale,\\
tout y parlerait\\
à l'âme en secret\\
sa douce langue natale.\\
\\
\tabto{2cm} là, tout n'est qu'ordre et beauté\\
\tabto{2cm} luxe, calme et volupté\\
\\
vois sur ces canaux\\
dormir ces vaisseaux\\
dont l'humeur est vagabonde;\\
c'est pour assouvir\\
ton moindre désir\\
qu'ils viennent du bout du monde.\\
\\
les soleils couchants\\
revêtent les champs,\\
les canaux, la ville entière\\
d'hyacinthe et d'or\\
le monde s'endort\\
d'une chaude lumière\\
\\
\tabto{2cm} là, tout n'est qu'ordre et beauté\\
\tabto{2cm} luxe, calme et volupté\\
}


% ====================================================================== %
\poem{3}{parfums exotiques}{\isPayant}{
quand, les deux yeux fermés, en un soir chaud d'automne\\
je resprire l'odeur de ton sein chaleureux\\
je vois se dérouler les rivages heureux\\
qu'éblouissent les feux d'un soleil monotone\\
\\
une île paresseuse où la nature donne\\
des arbres singuliers et des frutis savoureux\\
des hommes dont le corps est mince et vigoureux\\
et les femmes dont  l'\oe{}il par sa franchise étonne.\\
\\
guidé par ton odeur vers de charmants climats\\
je vois un port peuplé de voilets et de mâts\\
encor tout fatigué par la vague marine,\\
\\
pendant que le parfum des verts tamariniers\\
qui circule dans l'air et m'enfle la narine\\
se mêle dans mon âme au chant des mariniers.
}{}

% ====================================================================== %
\poemBreakCover{3}{enivrez-vous!}{\isPayant}{
il faut toujours être ivre.\\
tout est là: c'est l'unique question.\\
pour ne pas sentir l'horrible fardeau\\
du temps qui brise vos épaules\\
et vous penche vers la terre,\\
il faut vous enivrer sans trêve.\\
\\
mais de quoi? de vin, de poésie ou de vertu,\\
à votre guise. mais enivrez-vous.\\
\\
et si quelques fois, sur les marches d'un palais\\
sur l'herbe verte d'un fossé\\
dans la solitude morne de votre chambre,\\
vous vous réveillez, l'ivresse déjà diminuée ou disparue\\
demandez au vent, à la vague, à l'étoile,\\
à l'oiseau, à l'horloge, à tout ce qui fuit,\\
à tout ce qui gémit, à tout ce qui roule,\\
à tout ce qui chante, à tout ce qui parle\\
demandez quelle heure il est;\\
et le vent, la vague, l'étoile, l'oiseau, l'horloge vous répondront:
}{}

\poemBreakEnd{
\begin{flushright}
il est l'heure de s'enivrer!\\
pour n'être pas\\
les esclaves marthyrisés du temps,\\
enivrez-vous sans cesse!\\
de vin, de poésie ou de vertu,\\
à votre guise.
\end{flushright}
}


% ====================================================================== %
\poemBreakCover{4}{le lac}{\isPayant}{
Ainsi, toujours poussés vers de nouveaux rivages,\\
Dans la nuit éternelle emportés sans retour,\\
Ne pourrons-nous jamais sur l'océan des âges\\
Jeter l'ancre un seul jour?\\
\\
Ô lac! l'année à peine a fini sa carrière,\\
Et près des flots chéris qu'elle devait revoir,\\
Regarde! je viens seul m'asseoir sur cette pierre\\
Où tu la vis s'asseoir!\\
\\
Tu mugissais ainsi sous ces roches profondes,\\
Ainsi tu te brisais sur leurs flancs déchirés,\\
Ainsi le vent jetait l'écume de tes ondes\\
Sur ses pieds adorés.\\
\\
Un soir, t'en souvient-il? nous voguions en silence;\\
On n'entendait au loin, sur l'onde et sous les cieux,\\
Que le bruit des rameurs qui frappaient en cadence\\
Tes flots harmonieux.\\
\\
Tout à coup des accents inconnus à la terre\\
Du rivage charmé frappèrent les échos;\\
Le flot fut attentif, et la voix qui m'est chère\\
Laissa tomber ces mots:}{Alphonse de Lamartine, Méditations poétiques, 1820}

\poemBreakMiddle{
\og{}Ô temps! suspends ton vol, et vous, heures propices!\\
Suspendez votre cours:\\
Laissez-nous savourer les rapides délices\\
Des plus beaux de nos jours!\\
\\
\og{}Assez de malheureux ici-bas vous implorent,\\
Coulez, coulez pour eux;\\
Prenez avec leurs jours les soins qui les dévorent;\\
Oubliez les heureux.\\
\\
\og{}Mais je demande en vain quelques moments encore,\\
Le temps m'échappe et fuit;\\
Je dis à cette nuit: Sois plus lente; et l'aurore\\
Va dissiper la nuit.\\
\\
\og{}Aimons donc, aimons donc! de l'heure fugitive,\\
Hâtons-nous, jouissons!\\
L'homme n'a point de port, le temps n'a point de rive;\\
Il coule, et nous passons! "
}

\poemBreakMiddle{
Temps jaloux, se peut-il que ces moments d'ivresse,\\
Où l'amour à longs flots nous verse le bonheur,\\
S'envolent loin de nous de la même vitesse\\
Que les jours de malheur?\\
\\
Eh quoi! n'en pourrons-nous fixer au moins la trace?\\
Quoi! passés pour jamais! quoi! tout entiers perdus!\\
Ce temps qui les donna, ce temps qui les efface,\\
Ne nous les rendra plus!\\
\\
Éternité, néant, passé, sombres abîmes,\\
Que faites-vous des jours que vous engloutissez?\\
Parlez: nous rendrez-vous ces extases sublimes\\
Que vous nous ravissez?\\
\\
Ô lac! rochers muets! grottes! forêt obscure!\\
Vous, que le temps épargne ou qu'il peut rajeunir,\\
Gardez de cette nuit, gardez, belle nature,\\
Au moins le souvenir!\\
\\
Qu'il soit dans ton repos, qu'il soit dans tes orages,\\
Beau lac, et dans l'aspect de tes riants coteaux,\\
Et dans ces noirs sapins, et dans ces rocs sauvages\\
Qui pendent sur tes eaux.
}

\poemBreakEnd{
Qu'il soit dans le zéphyr qui frémit et qui passe,\\
Dans les bruits de tes bords par tes bords répétés,\\
Dans l'astre au front d'argent qui blanchit ta surface\\
De ses molles clartés.\\
\\
Que le vent qui gémit, le roseau qui soupire,\\
Que les parfums légers de ton air embaumé,\\
Que tout ce qu'on entend, l'on voit ou l'on respire,\\
Tout dise: Ils ont aimé!
}


% ====================================================================== %
\poem{5}{souffle de vie}{\isPayant}{
à toi le cynique, ô toi l'étranger\\
qui toutes ces années ne jura que par la vérité\\
cachait le soleil, éteignait l'arc-en-ciel\\
pariat d'un jeunesse qui te semblait lointaine\\
\\
tu fuyais la vie, tu fuyais la oie\\
ton plus grand porjet était d'avoir moins d'émoi\\
et de travailler, produire de la connaissance\\
car il n'y qu'ici que tu ne pensais avoir de la valeur\\
sur un banc d'école, ne faire que des sciences\\
tu te complaisais dans un bien triste bonheur.\\
\\
mais mon pauvre amis, toi qui te croît si grand\\
de vivre dans la misère n'es-tu point conscient?\\
de tant d'espériences tu es passé à côté\\
et dans ton lit tu repenses à toutes ces années gâchées.\\
\\
l'homme a tué Dieu [..] et c'est désormais à lui
que tu te repentis.\\ %
{[..]}
ces chaînes intérieur qui te transformèrent en cimetière\\ 
{[..]}
à ceux qui dans ton c\oe{}ur sont inscrits à l'encre de la vie.
}{}

% ====================================================================== %
\poem{5}{parenthèse}{\isPayant}{
{[..]} le rond de tes mots boucle encore dans ma tête\\
{[..]} de ton nom naissent des courbes aux formes indiscrètes\\
{[..]} mes doigts timides cherchent à retenir ton ombre\\
mais tu m'échappes toujours, sans jamais prévenir.
}{Constance Gires, Hiatus, 2022}

% ====================================================================== %
\poem{5}{titano}{\isPayant}{
voici quelques vers\\
dans ce lieu austère\\
mélodie du monastère\\
\\
la frayeur te frappe\\
alors tu t'échappes\\
de ce gouffre qui te hape\\
\\
j'ignore la suite\\
ma crainte est fortuite\\
pourtant je prends la fuite
}{}


% ====================================================================== %
\poem{5}{ivre de la vie}{\isPayant}{
Je suis né dans un beau jardin\\
seul, avec mon c\oe{}ur et mon destin\\
destiné à vivre dans l'errance\\
je vagabondais avec mes yeux d'enfance
}{Léo le Bouquin, Hiatus, 2022}

% ====================================================================== %
\begin{comment}
\poem{5}{vaincre le cahot}{\isPayant}{
souviens toi que les cahots de ton c\oe{}ur\\
sont ceux de la route vers le bonheur\\
\\
souviens toi que le vide destructeur\\
est bien moins fort que l'âme du rêveur
}{Victoire, Hiatus, 2022}
\end{comment}


% ====================================================================== %
\poem{5}{Fleuve}{\isPayant}{
au delà des frontières du monde et sa luxure\\
le soleil engourdi se fond dans un murmure\\
aux vacillements flasques d'une eau sombre et brisée\\
zébrures irisées, vocalise embrasée.\\
\\
clepsydre dyslexique, éclat récalcitrant\\
mélodrame éclectique éclipse un psaume errant\\
``crépuscule est pustule, pistil pestilentiel!''\\
homoncule crédule se croyant essentiel\\
\\
vomit un flot aride, ridicule et cupide\\
dans l'atmosphère humide, vapeur d'éther perfide\\
déconstruction précipitée, identité dénaturée\\
l'individu ectoplasmique quitte sa vie fantomatique.
}{Lorcha, Hiatus, 2023}


% ====================================================================== %
\poem{5}{Troie}{\isPayant}{
{[..]}\\
de l'immense palais qui me servait de crâne\\
il ne rete qu'une fleur, qui aujourd'hui se fane.\\
\\
{[..]}\\
debout dans ce désert que tu offres à ma vue\\
ce que jamais encore, mon âme avait souffert.\\
\\
{[..]}\\
\textit{le sens n'est plus un mot}, qui autrefois ciment\\
des briques de mon être le rendait cohérent\\
\textit{mais une fine brume}, qui recouvre le sol\\
aux reflets bleux, moirés comme une vapeur d'alcool\\
\textit{et empêche de voir}, clairement l'étendue\\
des ravages que tu fis, de tout ce que je fus\\
\textit{les vagues de mon c\oe{}ur} s'écrasant à tes pieds\\
l'écume sur ta peau laisse traces salées\\
\textit{et les plages de ton âme}, le long desquelles crèvent\\
sans cesse mes espoirs, sont de funestes grèves.
}{Léo Guy, Hiatus, 2023.}



% ====================================================================== %
\poem{6}{Silence et joie}{\isPayant}{
{[..]} ocre crépi, noyé de soleil\\
{[..]}\\
cette nuit, dame Lune, dans toute sa splendeur, dans toute sa rondeur {[..]} je
lui demanderai d'aller au bout du monde sauver la vie d'un homme amputé de la
parole divine.\\
{[..]}\\
un gentil fantôme. unve vieille dame tout blanche et souriante.
}{Nathalie Cohen, Happinez, 2022}


% ====================================================================== %
\poemBreakCover{6}{merci}{\isPayant}{
il est des êtres qui sèment les grains de nos rêves\\
du bout des lèvres, du fond de l'âme, sans trève\\
légère trainée de lune au monde de nos plumes\\
{[?]} dans la joie aux confins de la brume\\
\\
il est des êtres qui danses sur les peines les plus folles\\
embruns gris de la nuit entravés de licols\\
habillés des splendeurs que seule la tristesse donne\\
aux sourires invisibles qui doucement rayonnent\\
\\
il est des êtres qui savent envelopper de bonheur\\
du bout des yeux, miroirs délicats de nos heures\\
dédale ailé des plaines aux rivières amies\\
faune de nos amours colorée de la vie
}{}

\poemBreakEnd{
il est des êtres qui savent ouvrir leur c\oe{}ur et sang\\
sensibles aux tambours rouges de nos derniers instants\\
lumière des ondes du temps qui balbutie toujours\\
joueurs de terre et d'ambre, merveilleux troubadours\\
\\
il est des êtres qui sèment les graines de nos rêves\\
fleurs de rires enchantés qui jamais ne s'achèvent\\
oh! vous amis du vent, oh! vous amis du temps,\\
gratitude s'envole sur les ailes du printemps.\\
}



% ====================================================================== %
\poemBreakCover{7}{\nothing}{\isPayant}{
nous sommes les écrivains du couteau\\
nous sommes les penseurs de la panse\\
les savants de la croute de pain\\
les peintre de la suie\\
\\
nous sommes les prophètes des culottes sales\\
nous sommes les amateurs de l'estomac\\
les amants de descentes de gouttières\\
les comptables des charcas et des corneilles\\
\\
nous sommes les violonneux du mal de dents\\
nous sommes les amoureux du rhume\\
les goinfres de l'année passée\\
les ivrognes du jour d'hier\\
}{auteur russe}

\poemBreakEnd{
nous sommes des marchants des yeuxs noirs du ciel\\
nous sommes les richats aux écus jaunes sur l'ambre\\
les prêtres du rire aux éclats\\
les richards de l'aube\\
\\
des enfants de Dieu\\
Nous sommes tous $\qquad$ tous $\qquad$ aujourd'hui des tsars!
}
