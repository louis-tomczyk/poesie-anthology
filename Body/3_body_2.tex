\myPageBreak

% ====================================================================== %
\poem{?}{\nothing}{\isPayant}{
amis à vie, à mort\\
je t'emmènerai à\\
Miami, amor\\
\\
tu vaux plus que\\
mille sweets ami\\
qu'une suite au Ritz\\
plus qu'un Dali\\
\\
et le temps passe vite\\
alors on s'en ira\\
ailleurs
}{}

% ====================================================================== %
\poem{9}{\nothing}{\isPayant}{
quand je me sens des plis amers autour de la bouche,\\
quand mon âme est un bruineux et dégoulinant novembre,\\
et surtout lorsque mon cafard prend tellement le dessus\\
que je dois me tenir à quatre pour ne pas descendre dans la rue\\
y envoyer valdinguer le chapeau des gens,\\
je comprends qu'il est grand temps de prendre le large\\
ça remplace pour moi le suicide.
}{Herman Melville, Moby Dick: trouvé dans \OG{Écrivains voyageurs, ces vagabonds
qui disent le monde}. Laurent Maréchaux}

% ====================================================================== %
\poem{10}{que la blessure se ferme}{\isPayant}{
le silence de l'aimée\\
est un meurtre tranquille\\
il blesse sans tuer\\
il inquiète et fait monter la fièvre\\
c'est un mur froid qui avance\\
broie ce qu'il rencontre\\
le tout sans faire de bruit
}{Tahar Ben Jelloun, trouvé dans \OG{anthologie de la poésie française du XXe
siècle}}

% ====================================================================== %
\poem{10}{\nothing}{\isPayant}{
lumière absolue\\
feu blanc et origine de la question\\
à l'intérieur de la source\\
\\
traverser le mur\\
atteindre la niche\\
et ses ablutions avec la pierre du temps
}{}

% ====================================================================== %
\poemBreakCover{11}{\nothing}{\isPayant}{
si tu pars avec moi\\
les gens te montreront du doigt\\
si je pars avec toi, j'oublierai\\
qui j'étais\\
\\
tu seras hors la loi\\
si je t'enchaîne à moi\\
tu aimeras les chaînes\\
je m'accrocherai à toi\\
comme le lière d'un chêne\\
\\
l'endroit où l'on pourra vivre désespérement libre\\
on nous prend pour des fous\\
ce qu'on peut penser de nous on s'en fout\\
on se fout de toute\\
on se fout d'être malheureux\\
on s'aime encore mieux\\
quand on a plus rien à perdre\\
quand on a plus rien à perdre
}{un frère}

\poemBreakEnd{
j'ai la tête qui éclate\\
je voudrais seulement dormir\\
m'étendre sur la pierre\\
pour me laisser mourir\\
\\
quand on choisit sa vie\\
il faut la vivre jusqu'au bout\\
venez avec nous risquer nos vies\\
sur les autoroutes de la folie
}

% ====================================================================== %
\poem{12}{l'autre dieu}{\isPayant}{
déjà il se trouvait\\
au point de départ,\\
regardant la ligne de l'avenir\\
mails il a tranquillement fait demi-tour\\
et il est entré dans l'immense muraille\\
\\
cela arrive,\\
impossible d'achever\\
ce que nous n'avons pas fait\\
cela demeure à jamais\\
clairière de silence\\
dans la forêt des cris\\
rien n'est uniquement soi\\
ni la défaite, ni cette brumeuse victoire\\
ni le chagrin, qui élargit votre séjour\\
\\
ce qui n'a jamais eu lieu console\\
et nous supportons l'histoire\\
cet horrible récit\\
d'un autre avenir
}{poésies de finlande, Gösta Ågren, dans Runoja/Finsk Lyrik,
éditions \OG{le temps parallèle}}

% ====================================================================== %
\poem{12}{l'amour du pays}{\isPayant}{
le jours s'éclaircit, les troncs d'arbres\\
disparaissent dans un immense\\
nuage de feuilles, j'ai vu un papillon,\\
j'ai fait comme s'il existait.\\
\\
une main que je venais seulement de remarquer\\
m'a lâché, fermant les profondeurs\\
de sa demeure. seul, celui\\
qui habite une maison\\
peut choisir de s'en aller sur les routes\\
nous qui sommes sans demeure\\
nous n'avons rien à quitter\\
abrupte comme un cri, notre vie,\\
ici, où un jour a existé\\
un papillon
}{}

% ====================================================================== %
\poem{12}{moment}{\isPayant}{
je me souviens des morts\\
sans chagrin, forêt\\
abattue. je vois blanchir\\
quelques visages, ils deviennent\\
des oiseaux et se détachent\\
de corps obscurs. ils se rapprochent\\
doucement: ils ont un message pour moi\\
qui ne sera jamais transmis car l'existence\\
du message compte plus que le message. le vent\\
se réveille, il apaise le silence; le regard\\
s'élargit, devient espace. ainsi s'effacent-ils tous\\
tout simplement et les vagues\\
tantôt travail tantôt repos\\
battent la falaise
}{}

% ====================================================================== %
\poemBreakCover{12}{tristesse}{\isPayant}{
quelle tristesse\\
quelle obscurité autour de mon âme\\
comme un soir en automne dans un pays désert\\
inutile ici-bas toute peine\\
inutile toute lutte\\
inutile tout l'univers\\
\\
le ciel\\
je n'en veux pas, ni du noir de la géhenne\\
non plus d'une jeune fille dans mes bras\\
que mon destin soit:\\
échapper à la douleur de savoir\\
que tout me devienne vide, muet\\
\\
écoutez amis!\\
une dernière fois je vous le demande\\
écoutez, je vous en supplie\\
dans la maison de la mort, une chambre\\
que je puisse y habiter,\\
descendre au sein de la terre
}{Aleksis Kivi}

\poemBreakEnd{
une tombe pour moi\\
creusez une tombe à l'ombre des saules\\
et d'une couverture noire couvrez-la\\
ensuite pour toujours\\
quittez mon domaine\\
je veux reposer\\
en paix\\
\\
que jamais\\
ne s'élève un tertre\\
sur ma tombe\\
mais que l'argile se durcisse en pré\\
et que nul ne sache\\
que mon lieu de repos\\
se trouve sous le saule éteint
}

% ====================================================================== %
\poem{13}{Il pluere dans mon c\oe{}ur}{\isPayant}{
il pleure dans mon c\oe{}ur\\
comme il pleut sur la ville\\
quelle est cette langueur\\
qui pénètre mon c\oe{}ur?\\
\\
ô bruit doux de la pluie\\
par terre et sur les toîts!\\
pour un c\oe{}ur qui s'ennuie\\
ô le chant de la pluie!\\
\\
il pleure sans raison\\
dans le c\oe{}ur sui s'éc\oe{}ure.\\
quoi! nulle trahison?...\\
ce deuil est sans raison\\
\\
c'est bien la pire peine\\
de ne savoir pourquoi\\
sans amour et sans haine\\
mon c\oe{}ur a tant de peine
}{Paul Verlaine, Romances sans paroles, 1874}

% ====================================================================== %
\poem{14}{\nothing}{\isPayant}{
tu m'as fabriquée. elle n'existe pas,\\
cette femme là\\
point de pareilles sur la terre\\
aucun médecin pour te guérir,\\
aucun poète pour t'apaiser\\
c'est Dieu qui t'aidera à m'oublier\\
\\
nous nous sommes rencontrés\\
en une année de deuil\\
le monde avait perdu ses forces\\
tout ployait sous le poids du malheur\\
les tombes seules étaient fraîches\\
\\
sans réverbères, les flots de la Néva,\\
suie noire, coulaient\\
la nuit opaque se dressait comme un mur...\\
c'est alors que je t'ai appelé\\
je ne savais pas ce que je faisais\\
tu es venu, comme guidé par un étoile,\\
traversant cet automne tragique,\\
jusque dans cette \OG{maison à jamais dévastée}\\
d'où se sont envolés tous mes vers torturés.
}{Anna Akhmatova, L'églantier fleuri, 18/08/1956}

% ====================================================================== %
\poem{14}{\nothing}{\isPayant}{
le lien souvent se rompt\\
qui nous fait vibrer dès l'aurore\\
Dieu et les choses\\
en un instant nous échappent\\
le langage apparaît\\
comme un ornement du néant\\
une preuve inattaquable\\
de notre incapacité à demeurer\\
dans une soumission parfaite\\
au plus simple\\
nous perdons le pouvoir de songer
}{Pierre Oster, ibid}
