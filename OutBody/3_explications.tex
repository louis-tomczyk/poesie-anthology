

\begin{center}
    \vspace*{\fill}
    \begin{minipage}{0.8\linewidth}
   

\begin{itemize}[label = $\bullet$]
    \item Lorsque des mots étrangers sont utilisés,
        j'ai opté pour une transcription phonétique afin de
        tenter de me rapprocher autant que faire se peut
        de la prononciation de la langue d'origine.
        Le cas échéant un glossaire en fin fait le lien avec
        la translittération française, et est notifiée par la
        présence d'une petite épée.
        De même avec des figures ou des références non
        communes.
        
        $$\dagger$$

    \ifnum\isPayant = 1
        \item La mise \textbf{en gras} des dates indique la présence
            d'une note explicative et/ou contextuelle en fin de
            recueil.
    \fi
\end{itemize}


    \end{minipage}
    \vspace*{\fill}
\end{center}
\myPageBreak
